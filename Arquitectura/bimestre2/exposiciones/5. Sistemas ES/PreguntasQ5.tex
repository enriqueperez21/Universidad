% Created 2024-08-07 mié 23:32
% Intended LaTeX compiler: pdflatex
\documentclass[presentation]{beamer}
\usepackage[utf8]{inputenc}
\usepackage[T1]{fontenc}
\usepackage{graphicx}
\usepackage{grffile}
\usepackage{longtable}
\usepackage{wrapfig}
\usepackage{rotating}
\usepackage[normalem]{ulem}
\usepackage{amsmath}
\usepackage{textcomp}
\usepackage{amssymb}
\usepackage{capt-of}
\usepackage{hyperref}
\usetheme{default}
\usecolortheme{}
\usefonttheme{}
\useinnertheme{}
\useoutertheme{}
\author{Enrique Perez S}
\date{<2024-07-2>}
\title{Preguntas E/S Programada}

\hypersetup{
 pdfauthor={Enrique Perez S},
 pdftitle={Preguntas E/S Programada},
 pdfkeywords={},
 pdfsubject={},
 pdfcreator={Emacs 27.1 (Org mode 9.3)}, 
 pdflang={Spanish}}
\begin{document}

\maketitle
\begin{frame}{Outline}
\tableofcontents
\end{frame}


\section{Preguntas E/S Programada}
\label{sec:org787eb8d}
\begin{frame}[label={sec:org654cabf}]{Pregunta 1}
¿Qué es la E/S programada y cómo se diferencia de otros métodos de E/S?


Es un método de transferencia de datos entre el CPU y un dispositivo de E/S donde el CPU controla directamente las operaciones de E/S. Se diferencia de otros métodos como la E/S por interrupciones, donde el dispositivo notifica al CPU cuando está listo para transferir datos, y la E/S por acceso directo a memoria (DMA), donde un controlador especial maneja las transferencias de datos sin intervención constante del CPU.
\end{frame}

\begin{frame}[label={sec:org25e6a15}]{Pregunta 2}
¿Cuáles son las principales desventajas de la E/S programada en comparación con otros métodos de E/S?


Es algo ineficiente ya que el CPU debe esperar activamente (polling) hasta que el dispositivo de E/S esté listo para transferir datos, lo cual consume tiempo de CPU que podría ser utilizado para otras tareas. Además, puede llevar a una mayor latencia en la respuesta de las operaciones de E/S.
\end{frame}

\begin{frame}[label={sec:orgd53f750}]{Pregunta 3}
Describe el proceso de E/S programada paso a paso. ¿Qué roles juegan el CPU y los dispositivos de E/S en este proceso?


El CPU envía una instrucción al dispositivo de E/S para iniciar una operación.
El CPU consulta periódicamente el estado del dispositivo de E/S para verificar si está listo para transferir datos.
Una vez que el dispositivo está listo, el CPU transfiere los datos entre la memoria y el dispositivo.
El CPU repite este proceso hasta que la operación de E/S está completa.
El dispositivo de E/S se encarga de recibir o enviar datos según la instrucción del CPU, pero no puede notificar al CPU cuando está listo sin que el CPU lo consulte activamente.
\end{frame}

\begin{frame}[label={sec:org0108229}]{Pregunta 4}
¿Qué es un controlador de E/S y cuál es su función en la E/S programada?


E un hardware o dispositivo de software que gestiona las comunicaciones entre el CPU y los dispositivos de E/S. En la E/S programada, el controlador de E/S facilita la transferencia de datos y el control de las operaciones de E/S, proporcionando registros de estado y control que el CPU utiliza para iniciar y monitorizar las operaciones.
\end{frame}

\begin{frame}[label={sec:orgc9fcad0}]{Pregunta 5}
¿Cómo se gestionan los registros de estado y control en la E/S programada?


Los registros de estado y control son utilizados por el CPU para iniciar y monitorear las operaciones de E/S. El CPU escribe en los registros de control para indicar al dispositivo de E/S qué operación realizar y consulta los registros de estado para determinar si el dispositivo está listo para transferir datos.
\end{frame}

\begin{frame}[label={sec:org9e515d0}]{Pregunta 6}
¿Qué papel juega la sincronización en la E/S programada y cómo se asegura esta sincronización?


La sincronización es crucial en la E/S programada para asegurar que el CPU y el dispositivo de E/S estén listos para la transferencia de datos al mismo tiempo. Esta sincronización se asegura mediante el polling, donde el CPU verifica constantemente el estado del dispositivo de E/S para determinar cuándo puede transferir datos.
\end{frame}

\begin{frame}[label={sec:org6f3003d}]{Pregunta 7}
Explique el concepto de polling en el contexto de E/S programada. ¿Cuáles son sus ventajas y desventajas?

Polling es el proceso por el cual el CPU consulta periódicamente el estado de un dispositivo de E/S para verificar si está listo para transferir datos. La ventaja del polling es su simplicidad y control directo del CPU sobre las operaciones de E/S. La desventaja es que consume tiempo de CPU, ya que este debe esperar activamente la disponibilidad del dispositivo de E/S, reduciendo la eficiencia general del sistema.
\end{frame}

\begin{frame}[label={sec:org9a01568}]{Pregunta 8}
¿Cómo afecta la E/S programada el rendimiento general de un sistema informático?


Puede degradar el rendimiento del sistema porque el CPU debe dedicar tiempo a verificar continuamente el estado del dispositivo de E/S en lugar de realizar otras tareas. Esto puede llevar a una baja utilización del CPU y una mayor latencia en las operaciones de E/S.
\end{frame}

\begin{frame}[label={sec:orgb9ab827}]{Pregunta 9}
Describe un escenario práctico donde la E/S programada sería preferible sobre otros métodos de E/S.


Puede ser preferible en sistemas simples o embebidos donde el control directo y la simplicidad son más importantes que la eficiencia. Por ejemplo, en dispositivos donde las operaciones de E/S son poco frecuentes o donde el costo y la complejidad del hardware adicional para manejar interrupciones o DMA no están justificados.
\end{frame}

\begin{frame}[label={sec:org0f5ad86}]{Pregunta 10}
¿Qué técnicas se pueden emplear para mejorar la eficiencia de la E/S programada en un sistema?


Optimización del código de polling para reducir la frecuencia de verificación del estado del dispositivo.
Uso de temporizadores para espaciar las consultas del estado del dispositivo.
Implementación de buffers para manejar ráfagas de datos, permitiendo al CPU procesar datos en bloques en lugar de realizar transferencias individuales.
\end{frame}
\end{document}
